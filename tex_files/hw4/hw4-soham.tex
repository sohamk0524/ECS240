\documentclass[12pt]{article}
    \usepackage[margin=1in]{geometry}
    \usepackage{mdframed}
    \usepackage{subcaption}
    \usepackage{amssymb}
    \usepackage{amsmath}
    \usepackage{mathtools}

    % Custom colors
    \usepackage{color}
    \definecolor{deepblue}{rgb}{0,0,0.5}
    \definecolor{deepred}{rgb}{0.6,0,0}
    \definecolor{deepgreen}{rgb}{0,0.5,0}
    % Default fixed font does not support bold face
    \DeclareFixedFont{\ttb}{T1}{txtt}{bx}{n}{12} % for bold
    \DeclareFixedFont{\ttm}{T1}{txtt}{m}{n}{12}  % for normal
    \usepackage{listings}
    \usepackage{url}

    \usepackage{hyperref}

    \usepackage{booktabs}
    
    \usepackage{tikz}
    \usetikzlibrary{shapes,arrows,automata,positioning,cd}
    \tikzset{
      cfgedge/.style   = {black, ->, >=stealth},
      forward/.style = { blue, ->, >=angle 45},
      backward/.style = { red, densely dashed, ->, >=latex' },
      backwardleft/.style = { red, densely dashed, <-, >=latex' },
    }
    \usepackage{xcolor}
    
    \newcommand{\cfgarrow}{\mathbin{\tikz[baseline]\draw[cfgedge,yshift=0.6ex] (0,0) -- (.9em,0);}}
    \newcommand{\forwardarrow}{\mathbin{\tikz[baseline]\draw[forward,yshift=0.6ex] (0,0) -- (1em,0);}}
    \newcommand{\backwardarrow}{\mathbin{\tikz[baseline]\draw[backward,yshift=0.6ex] (0,0) -- (.95em,0);}}
  \newcommand{\dom}{\underline{\gg}}
    
    
    \begin{document}
    \lstset{
    language=C,
    basicstyle=\ttfamily\small,
    keywordstyle=\ttb\color{deepblue}\small,
    emph={foo,bar,assert,baz},          % Custom highlighting
    emphstyle=\ttb\color{deepred}\small,    % Custom highlighting style
    stringstyle=\color{deepgreen}\small,
    frame=tb,                         % Any extra options here
    numbers=left,
    stepnumber=1,
    showstringspaces=false            % 
    }
    
    \begin{center}
        \bigskip
        {\LARGE ECS 240 Programming Languages} \medskip
                
        {\Large Homework 4} \bigskip
    
    \end{center}
    
    \section*{About This Assignment}
    
    \begin{itemize}
      \item This assignment tests you on your understanding of Java points-to
      analysis.
      \item To complete the assignment 
            (i)~modify \texttt{hw4.tex}, 
            (ii)~create the corresponding PDF document (using pdflatex, for example), and 
            (iii)~submit the pdf electronically via Gradescope by the due date; 
            see \href{https://www.gradescope.com/get_started#student-submission}{this page} and
            \href{http://gradescope-static-assets.s3-us-west-2.amazonaws.com/help/submitting_hw_guide.pdf}{this document}.
            Your assignment will not be graded and you will receive
            no points if you do not follow these instructions. 
      \item \textbf{When submitting the assignment on Gradescope, please mark which page
        corresponds to each question on the assignment.} Your assignment will not be graded and you will receive
        no points if you do not follow these instructions. 
      \item The \LaTeX\ source has \texttt{TODO}~comments to clearly
        indicate where changes need to be made. 
      \item The \verb=\vspace= commands can be safely commented out.
      \item This assignment can be worked on in a group of at most three. Enter
      the names and email addresses of the team members in the space provided
      below.
    \end{itemize}

    \begin{mdframed}
      Team members:
      \begin{itemize}
        %TODO
        \item Soham Kolhatkar, sakolhatkar@ucdavis.edu % Enter name and email address of first team member.
        \item Divyansh Rajesh Jain, drajeshjain@ucdavis.edu % Comment out this line if working individually.
      \end{itemize}
    \end{mdframed}

    
    \newpage
    \begin{enumerate}

    

\item (5 points) Consider the following Java-like code:
  \begin{lstlisting}
    class MyInteger {
        public MyInteger(int v) {
            this.val = v;
        }
        public int val;
    }

    void bar (MyInteger a, MyInteger b) {
        if (a.val != b.val) return ;
        if (a.val < 50) return ;
        a.val--;
        b.val--;
        b.val--;
        // P1
    }
  \end{lstlisting}
  A program analyzer $A$
  reports the following expression as an invariant 
  at program point \lstinline$P1$: \lstinline$a.val > b.val$.

  Provide an input for \lstinline$bar$ which shows that analyzer $A$ is \emph{unsound}.
    \begin{mdframed}
  %\vspace{2em}
        \lstinline$MyInteger a = MyInteger(-2,147,483,647) //Integer.MIN_VALUE + 1; $
        
        \lstinline$MyInteger b = MyInteger(-2,147,483,647) //Integer.MIN_VALUE + 1;$

        \textbf{This answer is assuming that a.val and b.val are signed 32-bit integers}
    
        The example above shows an input that will violate the invariant at $P1$. This occurs in a bug because if you subtract 1 from $a.val$ then it will go to Integer.MIN\_VALUE, however if you subtract 2 from b, then it will result in an \textbf{integer underflow} and will result in $b.val$ being a positive number. This input shows that analyzer $A$ is unsound.
        
    \end{mdframed}

  \newpage
  \item (5 points) A Java statement involving multiple field reads and writes can be broken into
  multiple statements each with a single field read or write involving temporary variables.

  For example, the statement \lstinline$ x = y.field1.field2$ \\
  can be rewritten as \\ 
  \lstinline$T1 t1 = y.field1; x = t1.field2;$ where
  \lstinline$T1$ is the type of \lstinline$y.field1$.

  Similarly, the statement \lstinline$ b.field1.field2 = a;$ \\
  can  be rewritten as \\ 
  \lstinline$T1 t1 = b.field1; t1.field2 = a;$ where
  \lstinline$T1$ is the type of \lstinline$b.field1$.

  Given a Java program $P$, let $OneField(P)$ denote such a rewrite.
  This rewrite does not reduce the precision of flow-sensitive field-sensitive points-to analysis for Java
  (if you only consider the points-to sets of the variables in the original program $P$).
  
  State whether the following is \textbf{True} or \textbf{False}:

  \emph{The result of a precise flow-insensitive field-sensitive points-to analysis 
  for $P$ and $OneField(P)$ is the same with respect to the variables in 
  the program $P$.}

  If \textbf{True}, justify your answer.
  If \textbf{False}, provide a counter-example in the form of a program $P$,
  the rewritten program $OneField(P)$, and the points-to sets for variables
  in $P$ and $OneField(P)$.
    
  \begin{mdframed}
    %%\vspace{2em}
    %% TODO
    %% \textbf{True} Justify your answer.
    \textbf{False}

    \textbf{Program P}
        \begin{lstlisting}
          a1 = o1;
          b1 = o2;
          a.g = a1;
          b.g = b1;
          p.f = &a;
          p.f = &b;
          p.f = p.f.g;
        \end{lstlisting}
        %% Points-to relations for P using flow-insensitive analysis
        \textbf{Points-to relations for P Precise Flow-Insenstive Analysis}

        $\{\langle a1, o1 \rangle\, \langle b1, o2 \rangle\, \langle a, a1 \rangle, \langle b, b1 \rangle, \langle p, a \rangle, \langle p, b \rangle, \langle a, a2 \rangle, \langle b, b2 \rangle \}$
        %% Program OneStar(P)

        \textbf{Program OneField(P)}
        \begin{lstlisting}
          a1 = o1;
          b1 = o2;
          a.g = a1;
          b.g = b1;
          p.f = &a;
          p.f = &b;
          t1 = p.f
          t2 = t1.g
          p.f = t2
        \end{lstlisting}

                \textbf{Points-to relations for OneStar(P) Precise Flow-Insensitive Analysis}
        $\{\langle a1, o1 \rangle\, \langle b1, o2 \rangle\, \langle a, a1 \rangle, \langle b, b1 \rangle, \langle p, a \rangle, \langle p, b \rangle, \langle a, a2 \rangle, \langle b, b2 \rangle, \langle a, b2 \rangle, \langle b, a2 \rangle \}$

    
  \end{mdframed}

\end{enumerate}
    
\end{document}